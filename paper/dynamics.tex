\documentclass{sig-alternate}
\usepackage{graphicx}

\begin{document}
%
% --\item Author Metadata here ---
\conferenceinfo{WWW}{'15}
%\CopyrightYear{2007} % Allows default copyright year (20XX) to be over-ridden \item IF NEED BE.
%\crdata{0-12345-67-8/90/01}  % Allows default copyright data (0-89791-88-6/97/05) to be over-ridden \item IF NEED BE.
% --\item End of Author Metadata ---

\title{The Evolution of Micro-task Crowdsourcing}
\subtitle{The Case of Amazon Mechanical Turk}
% \title{The Dynamics of Micro-task Crowdsourcing -- The Case of MTurk}
% anatomy of a Micro-task Crowdsourcing platform
% unravelling micro-task crowdsourcing dynamics/processes
% the evolution of micro-task crowdsourcing


%\numberofauthors{1}
\author{}

\maketitle
\begin{abstract}

Micro-task crowdsourcing is gaining popularity among serval organizations and research corps to leverage Human Computation in their daily operations. Unlike any other ``technology'', a crowdsourcing platform is subject to many factors that affect the performance, both in terms of speed and quality, of its services. Indeed, such factors shape the \emph{dynamics} of the market. For example, a common result is that increasing the price of a HIT would lead to faster results, however, we still do not know the exact impact of changing the price in the presence of many reputable requesters on the platform. Or, what happen when you post a link of a batch on a popular crowdsourcing forum etc.

In this paper we adopt a data-driven approach to analyze the behavior of the main actors in MTurk micro-task crowdsourcing: workers, requesters, HITs, and platform (MTurk); thanks to a collection of datasets that we gathered during the last 5 year.

The ultimate contribution that we propose is a predictive model to derive the expected performance of a published batch of HITs on MTurk at a specific moment in time.

\end{abstract}

% A category with the (minimum) three required fields
\category{H.4}{Information Systems Applications}{Miscellaneous}
%A category including the fourth, optional field follows...
% \category{D.2.8}{Software Engineering}{Metrics}[complexity measures, performance measures]

\terms{Design, Experimentation, Human Factors}

\keywords{Crowdsourcing, social networks, log analysis}

%!TEX root = ../dynamics.tex
\section{Introduction}\label{sec:intro}
The micro-task crowdsourcing market has seen a rapid growth in the last five years. This is correlated with the fact that more data is available to enterprises and that data is an asset within business processes. While data is available, its quality is not always high and manual processing is often necessary. To this end, outsourcing  data processing tasks  like image tagging, audio transcription, translation, etc. to a large crowd of individuals has become popular.

%micro-task Crowdsourcing platform
To perform such Human Intelligence Tasks (HITs) crowdsourcing platforms have been developed. Such platforms serve as place where the crowd (i.e., people willing to perform small tasks in exchange of a small monetary reward, also know as \emph{workers}) and work providers (i.e., also known as \emph{requesters}) meet. \gd{describe crowdsourcing process of publishing tasks, completing task, getting results, rewarding}

%
In this paper we analyze the evolution of a very popular micro-task crowdsourcing platform over the last five years and propose techniques to support requesters in posting HITs to the market in an effective way. Using features derived from a large-scale analysis of platform logs, we propose methods to predict the throughput of the crowdsourcing platform for a batch of HITs published by a certain requester at a certain point in time. This prediction is based on different features including current platform load, requester reputation, popular task types, etc.

\gd{summarize main findings}


In summary, the main contributions of this paper are:
\begin{itemize}

	\item An analysis of the evolution of a popular micro-task crowdsourcing market looking at dimensions like topics, task types, reward, worker location.

	\item A large-scale classification of 2.5M HITs published on Amazon MTurk.

	\item An effective model to predict the completion speed of a batch published on Amazon MTurk at a certain point by a certain requester.

	\item An online service with evolution statistics and services for requesters to better design their HIT metadata (e.g., keywords, reward, etc.)

\end{itemize}


The rest of the paper is structured as follows.
In Section \ref{sec:relwork} we overview recent work on micro-task crowdsourcing specifically focusing on works improving efficiency of crowdsourcing systems.
%!TEX root = ../dynamics.tex
\section{Related Work}\label{sec:relwork}

\subsection{Micro-task Crowdsourcing}

A very popular platform is Amazon MTurk\footnote{\url{http://mturk.com}} which provides access to a crowd of workers distributed worldwide but mainly composed of people based in USA and India \cite{mturk}.

Workers share experiences about HITs and requesters within web forums and ad-hoc websites \cite{turkopticon}. Such requester `reviews' serve as a way to measure requester reputation which is assumed to be a reason for obtaining answers efficiently from the crowd \cite{}. In this work we experimentally show the effect of platform and requester properties on crowd efficiency.  \gd{rephrase last sentence?}


\subsection{Human Computation}
Micro-task crowdsourcing is used to improve the quality of purely machine-based systems in order to combine both the scalability of machines over large amounts of data as well as the quality of human intelligence to process and understand data.
Many examples of such hybrid approaches exist.
Crowd-powered databases \cite{crowddb} leverage crowdsourcing to deal with incomplete data, complex data integration problems, graph search, and joins \cite{crowder,graphsearch,crowdjoins}.
Semantic Web systems leverage the crowd for schema matching \cite{crowdmap}, entity linking \cite{zencrowd}, and ontology engineering \cite{bioonto}.
Information Retrieval systems use crowdsourcing for evaluation purposes \cite{mizzaroalonso}.

When building systems that build on top of crowdsourcing platforms, two main challenges have to be dealt with: effectiveness and efficiency  of the crowd.

\subsection{Crowdsourcing Effectiveness}

A very important dimension for crowdsourcing effectiveness is \emph{answer aggregation}, that is, after assigning the same HIT to multiple workers in the crowd, effectively aggregate their answers in a final results to be output back from the crowdsourcing platform to the system leveraging human computation. Multiple approaches have been proposed for this aspect of crowdsourcing quality (e.g.,
\cite{Venanzi:2014:CBA:2566486.2567989,square,zencrowd,Hosseini:2012:ALM:2260641.2260661}).
% 
A recent approach which results in effective aggregation is \cite{Venanzi:2014:CBA:2566486.2567989} where authors detect communities of workers with similar answering patterns to re-weight their answers even when little evidence of individual worker quality is available (i.e., just few HITs have been completed).


More than aggregation, identifying the best worker in the crowd is another mean to obtain quality answers from the crowd \cite{pickacrowd,bozzon}.

More recently, \cite{Jung14-hcomp} aims at predicting work quality looking at worker activities over time.

\subsection{Crowdsourcing Efficiency}

In \cite{Kittur:2013:FCW:2441776.2441923} authors provide their view on how the crowdsourcing market should adapt specifically focusing on how to  support full-time crowd workers. On the other hand, in this work we perform a data-driven analysis of the evolution of micro-task crowdsourcing over the past five years using the findings of such analysis as features to support requesters while publishing HITs on these platforms in an effective way.

\cite{finishthem,scaleup}

To improve human computation efficiency, it is key to support worker in being more efficient while working on HITs. For example, workers spend way too much time in searching for HITs to work on \cite{Kucherbaev:2014:TET:2598153.2602249}.






% final remark
Our work is complementary to existing work as we present a data-driven study of the evolution of micro-task crowdsourcing over five years as a support evidence of the ongoing efforts to improve crowdsourcing quality and efficiency that we have described.

\section{The Amazon Mechanical Turk Crowdsourcing Platform}

\section{The Evolution of MTurk}
2009-2014
\subsection{A Data-driven Analysis}
Datasets: tracker, to, 
\subsection{Why Did Certain Topics Became Popular?}
\subsection{Why Did Certain Countries Were Preferred?}
top keywords per country (over time)
\subsection{Why Did Certain Requesters Quit?}
\subsection{Does reputation Improve Throughput?}

\section{Predicting the Platform Throughput}
\begin{itemize}

	\item ML tasks: classification and regression

	\item ML results: accuracy and error

	\item Best features

\end{itemize}

\section{Conclusions}

\end{document}
