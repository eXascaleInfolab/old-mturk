\documentclass{sig-alternate}
\usepackage{graphicx}
\usepackage{url}
\usepackage{amssymb}
\usepackage[usenames,dvipsnames]{color}
\usepackage{enumitem}
\usepackage{float}

\usepackage{caption}
\usepackage{subcaption}
\usepackage{enumerate}

\newcommand{\gd}[1]{\textcolor{ForestGreen}{GD: #1}}
\newcommand{\amt}[1]{Amazon MTurk}

\begin{document}
%
% --\item Author Metadata here ---
\conferenceinfo{WWW}{'15}
%\CopyrightYear{2007} % Allows default copyright year (20XX) to be over-ridden \item IF NEED BE.
%\crdata{0-12345-67-8/90/01}  % Allows default copyright data (0-89791-88-6/97/05) to be over-ridden \item IF NEED BE.
% --\item End of Author Metadata ---

% \title{The Evolution of Micro-task Crowdsourcing Markets}
\title{The Dynamics of Micro-Task Crowdsourcing}
\subtitle{The Case of Amazon MTurk}
% The Evolution of Micro-Task Crowdsourcing Markets —- The Case of Amazon Mechanical Turk
% The Dynamics of Micro-Task Crowdsourcing -- The Case of Amazon MTurk
% Anatomy of a Micro-Task Crowdsourcing Platform
% Unravelling Micro-Task Crowdsourcing Dynamics/Processes
% The Market for HITs: Throughput Uncertainty and the Market Mechanism


\numberofauthors{1}
\author{
\alignauthor
Djellel E. Difallah$^*$, Michele Catasta$^\dagger$, Gianluca Demartini$^\ddagger$, \\  Panagiotis G. Ipeirotis$^\diamond$, Philippe Cudr\'e-Mauroux$^*$\\
       \affaddr{$^*$ eXascale Infolab, University of Fribourg, Switzerland}\\
       \affaddr{$^\dagger$ EFPL, Switzerland}\\
       \affaddr{$^\ddagger$ University of Sheffield, UK}\\
	   \affaddr{$^\diamond$ New York University, USA}
       % \email{trovato@corporation.com}
}

\maketitle
\begin{abstract}
Micro-task crowdsourcing is rapidly gaining popularity among research communities and businesses as a means to leverage Human Computation in their daily operations. Unlike any other service, a crowdsourcing platform is in fact a marketplace subject to human factors that affect its performance, both in terms of speed and quality. Indeed, such factors shape the \emph{dynamics} of the crowdsourcing market. For example, a known behavior of such markets is that increasing the reward of a set of tasks would lead to faster results. However, it is still unclear how different dimensions interact with each other: reward, task type, market competition, requester reputation, etc.

In this paper, we adopt a data-driven approach to (A) perform a long-term analysis of a popular micro-task crowdsourcing platform and understand the evolution of its main actors (workers, requesters, tasks, and platform). (B) We leverage the main findings of our five year log analysis to propose features used in a predictive model aiming at determining the expected performance of any batch at a specific point in time. We show that the number of tasks left in a batch and the time at which the batch was posted are two key features of the prediction. (C) Finally, we conduct an analysis of the demand (new tasks posted by the requesters) and supply (number of tasks completed by the workforce) and show how they affect task prices on the marketplace.
\end{abstract}

% A category with the (minimum) three required fields
\category{H.4}{Information Systems Applications}{Miscellaneous}
%A category including the fourth, optional field follows...
% \category{D.2.8}{Software Engineering}{Metrics}[complexity measures, performance measures]

\terms{Design, Experimentation, Human Factors}

\keywords{Crowdsourcing, trend identification, tracking, forecasting}

%!TEX root = ../dynamics.tex
\section{Introduction}\label{sec:intro}
% 
While general data availability increases, its quality is not necessarily perfect and manual data pre-processing is often necessary before using it to create value or to support decisions.
% 
To this end, outsourcing  data-processing tasks like, for example, image tagging, audio transcription, translation, etc. to a large crowd of individuals on the Web has become more popular over time.

%micro-task Crowdsourcing platform
To perform such Human Intelligence Tasks (HITs), crowdsourcing platforms have been developed. Such platforms serve as a place where the crowd (\emph{workers}) willing to perform small tasks (so called \emph{micro-tasks}) in exchange of a small monetary reward and work providers (also known as \emph{requesters}) meet. 

The micro-task crowdsourcing market has seen a rapid growth in the last five years. This is also explained by the fact that large amounts of data are today available in companies, which are increasingly seen as a key asset for optimizing all business processes.

% crowdsourcing process of publishing tasks, completing task, getting results, rewarding}
The  micro-task crowdsourcing process works as follows. First, the requesters design the HIT based on their data and required task. Next, they publish batches of HITs on the crowdsourcing platform specifying their requirements and the monetary amount rewarded to workers in exchange of the completion of each HIT. Then, the workers willing to perform the published HITs complete the tasks and submit their work back to the requester who obtains the desired results and pays workers accordingly.

%s
In this paper, we analyze the evolution of a very popular micro-task crowdsourcing platform (i.e., \amt{}\footnote{\url{http://mturk.com}}) over a  five-year time span and report key findings about how
the market behaves with regards to demand and supply.
% 
Using features derived from a large-scale analysis of the platform logs, we propose methods to predict the throughput of the crowdsourcing platform for a batch of HITs published by a given requester at a certain point in time. This prediction is based on different features including the current platform load, the task type, etc. Using this prediction method, we try to understand the impact of each feature that we consider, and its scope over time.

% summarize main findings}
The main findings of our analysis are: 1) the type of tasks published on the platform has changed over time with content creation HITs being the most popular today; 2) the HIT pricing approach evolved towards larger and higher paid HITs to better attract workers in a competitive market; 3) geographical restrictions are applied to certain task types (e.g., surveys for US workers); 4) we observe an organic growth in the number of new requesters who use the platform, which is a sign of a healthy market; 5) we identify \emph{size of the batch} as the main feature that impacts the progress of a given batch; 6) we observe that supply (the workforce) has little control over driving the price of demand.

In summary, the main contributions of this paper are:
\begin{itemize}[noitemsep,topsep=0pt,parsep=0pt,partopsep=0pt]

	\item An analysis of the evolution of a popular micro-task crowdsourcing platform looking at dimensions like topics, reward, worker location, task types, and platform throughput.

	\item A large-scale classification of 2.5M HIT types published on \amt{}.

	\item A predictive analysis of HIT batch progress using more than 28 different features.
	
	\item An analysis of the crowdsourcing platform as a market (demand and supply).
	
\end{itemize}


The rest of the paper is structured as follows.
In Section \ref{sec:relwork}, we overview recent work on micro-task crowdsourcing specifically focusing on how  efficiency and effectiveness  of crowdsourcing platforms are addressed.
Section \ref{sec:stats} presents how \amt{} has evolved over time in terms of topics, reward, and requesters.
Section \ref{sec:type} summarizes the results of a large-scale analysis on the types of HIT that have been requested and completed over time.
Based on the previous findings, Section~\ref{sec:throughput} presents our approach to predicting the throughput of the crowdsourcing platform for a batch of published HITs.
Section~\ref{sec:market} studies the \amt{} market and how different events correlate (e.g., new HITs attracting more workers to the platform).
We discuss our main findings in Section~\ref{sec:discuss} before concluding in 
Section~\ref{sec:conc}.


%!TEX root = ../dynamics.tex
\section{Related Work}\label{sec:relwork}

The objective of this work is to understand and characterize how a micro-task crowdsourcing platform behaves as a marketplace. Thus, we first start by reviewing related work on human computation and micro-task crowdsourcing, related market analysis work and other propositions on how to improve and build future platforms.

\paragraph{Micro-task Crowdsourcing}
Crowdsourcing is defined as the outsourcing of tasks to a crowd of individuals over the Web. Crowdsourcing has been used for a variety of purposes, from innovation to software development \cite{platforms}. 
% 
Early crowdsourcing examples leveraged the fun or community belonging incentives (e.g., Wikipedia) rather than the monetary one.
Examples systems based on gamification include the ESP game \cite{vonAhn:2008:DGP:1378704.1378719} where players must agree on tags to use for a picture without possibility to interact with each other. 
% 
An extension of the ESP game is Peekaboom: a game that asks players to annotate specific objects within an image \cite{vonAhn:2006:PGL:1124772.1124782}.

In this work, we focus on \emph{paid micro-task crowdsourcing}, where the crowd is asked to perform short tasks, also known as Human Intelligence Tasks (HITs), in exchange for a small monetary reward per unit. Popular examples of such tasks include: spell checking, sentiment analysis of tweets, sanitize product reviews, or transcription of scanned shopping receipts.

Micro-task crowdsourcing is often used to improve the quality of machine-run algorithms in order to combine both the scalability of machines over large amounts of data as well as the quality of human intelligence in processing and understanding data \cite{vonAhn:2008:DGP:1378704.1378719}. Many examples of such hybrid human-machine approaches exist.
% 
Crowd-powered databases \cite{crowddb} leverage crowdsourcing to deal with problems like  data incompleteness,  data integration, graph search, and joins \cite{crowder,graphsearch,crowdjoins}.
% 
Semantic Web systems leverage the crowd for tasks like schema matching \cite{crowdmap}, entity linking \cite{zencrowd}, and ontology engineering \cite{bioonto}.
% 
Information Retrieval systems have used crowdsourcing for evaluation purposes \cite{mizzaroalonso}.
% 
Models and paradigm for hybrid human-machine systems have been proposed on top of crowdsourcing platforms \cite{crowdcomputer}, also including the design of hybrid workflows \cite{workflows}.

In this work, we specifically focus on micro-task crowdsourcing and analyze the dynamics of a  very popular  platform for this purpose: \amt{}. This platform provides access to a crowd of workers distributed worldwide but mainly composed of people based in the US and India \cite{mturk}. Many \amt{} workers share their experience about HITs and requesters through dedicated web forums and ad-hoc websites \cite{turkopticon}. Requester `reviews' serve as a way to measure the reputation of the requesters among workers and it is assumed to influence the latency of the tasks published \cite{TOreputation}, as workers are naturally more attracted by HITs published by  requesters with a good reputation.


%%problems
%When creating systems that build on top of crowdsourcing platforms, two main challenges have to be dealt with: optimizing the \emph{effectiveness} and the \emph{efficiency}  of the crowd. When compared to machine-based algorithms, human workers perform several orders of magnitude slower. On the other hand, human individuals are capable of data processing that machines cannot do. Ensuring the quality of the answers coming from the crowd is also a concern: human workers can make mistakes and some of them intentionally do not complete HITs properly with the sole intent of obtaining the monetary reward attached to the HIT. Next, we discuss works on improving both crowdsourcing effectiveness and efficiency.


%%domains
%In the case of large enterprises,  knowledge is often distributed across a number of employees. Crowdsourcing within an enterprise (i.e., when the crowd is composed by company employees) is  becoming popular and can benefit from the fact that employees are domain experts and can solve tasks better and faster than anonymous crowds. In this case, crowdsourcing can be used, for example, to efficiently find solutions to operational issues  \cite{enterprisecrowdsourcing}. 
%% 
%Crowdsourcing has also been used in the biomedical domain where, for example,  ontological relations among diseases can be validated by the crowd \cite{bioonto,biomedical}.

Because of the complex mechanisms existing between the workers, the requesters, and the platform itself, characterizing the dynamics and evolution of micro-task crowdsourcing platforms is key in order to understand the impact of the various components and to design better human computation systems. The goal of our work is to understand the evolution over time of a micro-task crowdsourcing platform and to identify key properties that can be used as future platform requirements.

\paragraph{Market Analysis}
An initial work analyzing \amt{} market was done in \cite{mturk}, our paper extends on this work by considering the time dimension and analyze long term trends and changes.
Faradani et al. \cite{faradani2011s} proposed a model to predict the completion time of a batch. Our prediction endeavor is however different, in the sense that we aim at predicting the immediate throughput based on current market condition and  to understand what features are having more impact than others.

%\paragraph{Crowdsourcing Effectiveness}
%%aggregation
%A very important dimension for crowdsourcing effectiveness is \emph{answer aggregation}: After assigning the same HIT to multiple workers in the crowd, one needs to effectively aggregate their answers into a final result that is sent back from the crowdsourcing platform to the system leveraging human computation. Multiple approaches have been proposed for this aspect of crowdsourcing quality (e.g.,
%\cite{Venanzi:2014:CBA:2566486.2567989,square,zencrowd,Hosseini:2012:ALM:2260641.2260661}).
%% 
%A recent approach that results in effective aggregation of crowd answers is \cite{Venanzi:2014:CBA:2566486.2567989}, where authors propose methods to detect communities of workers with similar answering patterns in order to re-weight their answers even when little evidence on the individual worker quality is available (i.e., just few HITs have been completed).
%
%%expert finding
%More than just aggregating their answers,  being able to identify the best workers in the crowd is a different means to obtain quality answers from the crowd \cite{pickacrowd,bozzon}. 
%% 
%Recent approaches aim at predicting work quality looking at worker activities over time  \cite{Jung14-hcomp}.
%% 
%This leads to \emph{task allocation} or \emph{routing} approaches that aim at assigning HITs to  workers currently available on the platform \cite{goel2014mechanism,crowdstar}.
%
%%spam
%On the other hand, workers producing low-quality results should also be identified and filtered out from the crowdsourcing process.
%Detecting malicious workers who aim at gaining the monetary reward without honestly  completing the HITs is also an active domain of research \cite{collusion}.


%\paragraph{Crowdsourcing Efficiency}
%While humans are not able to process data as fast as machines, different techniques and incentives can be used to improve work efficiency in a crowdsourcing setting.
%% 
%One way to improve human computation efficiency is by supporting workers in being more efficient while working on HITs. For example, workers spend way too much time in searching for HITs to work on \cite{Kucherbaev:2014:TET:2598153.2602249}. Thus, the task allocation and routing approaches described above can be beneficial for efficiency as well.
%% 
%Another angle on crowdsourcing efficiency is the use of HIT \emph{pricing schemes}. For example, in \cite{finishthem} authors propose models to set the HIT reward given some latency and budget constrains. In \cite{scaleup}, we studied how worker retention can improve the latency of a batch by leveraging varying bonus schemes.

\paragraph{Improving Future Crowdsourcing Platforms}
In \cite{Kittur:2013:FCW:2441776.2441923} authors provide their view on how the crowdsourcing market should evolve in the future, specifically focusing on how to support full-time crowd workers. 
Likewise, our goal is to identify ways of improving crowdsourcing marketplaces by understanding the dynamics of such platforms---based on historical data and models.
% final remark

Work on novel crowdsourcing platforms has  proposed methods for identifying the best workers in the crowd for specific tasks \cite{pickacrowd,bozzon}. Given the diverse task types being published in micro-task crowdsourcing platforms, such functionalities could be used to improve the work experience of the crowd and the quality of results obtained by requesters.

A way to improve crowdsourcing efficiency is the use custom HIT \emph{pricing schemes}. For example, in \cite{finishthem} authors propose models to set the HIT reward given some latency and budget constrains. In \cite{scaleup}, we studied how worker retention can improve the latency of a batch by leveraging varying bonus schemes.

Our work is complementary to existing work as we present a data-driven study of the evolution of micro-task crowdsourcing over five years.
%Our findings can be used as support evidence to the ongoing efforts in improving crowdsourcing quality and efficiency that are described above.
Our work can be also used to support requesters in publishing HITs on these platforms and getting results more efficiently.








%!TEX root = ../dynamics.tex
\section{The Evolution of Amazon MTurk: 2009 to 2014}\label{sec:stats}

\subsection{Crowdsourcing Platform Dataset}\label{sec:tracker}
Over the past five years, we have periodically collected data about HITs being published on \amt\ .
Real-time data collected from the crowdsourcing platform is available at \url{http://mturk-tracker.com/}.
Interactive visualizations of the historical data are available at \url{http://xi-lab.github.io/mturk-mrkt/}.

\gd{add basic data stats, like number of hits, requesters, etc}

\subsection{A Data-driven Analysis of Platform Evolution}
\paragraph{Popular Topics  Over Time}
First we want to understand how different topics have been addressed by means of micro-task crowdsourcing over time.
In order to do such analysis we look at the tags associated with published HITs. We observe the evolution of tag popularity and associated reward on \amt\ . 
%plot explanation
Figure \ref{fig:tagEvolution} shows such behavior. Each point in the plot represent a tag associated to HITs with its frequency (i.e., number of HITs with such tag) on the x asis, its average reward of HITs with this tag on the y axis in a certain year. The path connecting data points indicates the time evolution, starting in 2009, with one point representing the tag usage over one year.
% observations
We can observe that `audio' and `transcription' tags (i.e., blue and red path from left to right) have substantially increased in frequency over time being among the most popular tags in the last two years and are paid more than \$1 on average.
HITs with the `video' tag has also increased in number with a reward that has reached a peak in 2012 and decreased after that.
HITs tagged as `categorization' have been paid constantly in the range \$0.10-\$0.30 on average, except in 2009 where they were rewarded less than \$0.1.
HITs tagged as `tweet' have not increased in number but have been paid more over years reaching \$0.9 in 2014: This can be explained by more complex tasks being requested to workers: for example, from sentiment classification to writing a tweet.

\begin{figure}[htbp]
	\centering
		\includegraphics[width=0.5\textwidth]{figures/tagEvolution}
	\caption{tagEvolution}
	\label{fig:tagEvolution}
\end{figure}

\paragraph{Country Preference by Requesters Over Time}
Figure \ref{fig:country} shows the requirements of requesters with respect to which countries they need workers from. The left part of Figure \ref{fig:country} shows that most HITs are to be completed exclusively by workers located in US, India, or Canada. The right part of Figure \ref{fig:country} shows the evolution over time of the country requirement phenomena.
The plot shows the number of HITs with a certain country requirement (on the y axis) and its time evolution (on the x axis) with yearly steps. The size of the data point indicates the total reward associated to those HITs.
We can see that US-only HITs dominate both in terms of number as well as of reward associated to them. 
Interestingly, we notice how over time HITs uniquely for workers based in India have been decreasing. 
While HITs for workers based in Canada have been increasing over time being in 2014 more than those for workers based in India, we see that the reward associated to them is less as compared to the budget for Indian-only HITs.
As of 2014, both HITs for workers based in Canada or UK are more numerous that those for workers based in India.

\gd{how many HITs have country filters and how many have no country filter in percentage?} 

\begin{figure*}[htbp]
	\centering
		\includegraphics[width=0.48\textwidth]{figures/map}
		\includegraphics[width=0.48\textwidth]{figures/countriesTime}
	\caption{HITs with specific country requirements. On the left, the countries with most HITs dedicated to them. On the right, the time evolution (x-axis) of country specific HITs with volume (y-axis) and reward information (size of data point).}
	\label{fig:country}
\end{figure*}


% \subsection{Why Did Certain Requesters Quit?}
% \subsection{Does reputation Improve Throughput?}

\paragraph{HIT Reward Analysis}
\begin{figure}[htbp]
	\centering
		\includegraphics[width=0.5\textwidth]{figures/reward_year}
	\caption{Reward \gd{can we add 2014?}}
	\label{fig:reward_year}
\end{figure}

\begin{figure}[htbp]
	\centering
		\includegraphics[width=0.5\textwidth]{figures/requesters_reward}
	\caption{Reward and Requesters}
	\label{fig:requesters_reward}
\end{figure}



\begin{figure}[htbp]
	\centering
		\includegraphics[width=0.5\textwidth]{figures/keywords_location}
	\caption{Keywords vs Location}
	\label{fig:keyword_loc}
\end{figure}

\begin{figure}[htbp]
	\centering
		\includegraphics[width=0.5\textwidth]{figures/batch_size}
	\caption{Batch Sizes}
	\label{fig:batch_size}
\end{figure}



%!TEX root = ../dynamics.tex
\section{Large-Scale HIT Type Analysis}\label{sec:type}

\subsection{HIT Types}
\begin{itemize}

	\item CA

	\item CC

	\item IA

	\item IF

	\item SU

	\item VV

\end{itemize}

\subsection{Classification of HITs}
\begin{itemize}

	\item sample of 5000 HITs with type labelled by means of crowdsourcing. We asked workers on MTurk to assign a HIT to one of the defined classes by presenting them with the title, description, keywords, reward, allotted time for the HIT. The instructions contained a definition and examples for each task type.

	\item After assigning each HIT to three different workers in the crowd, a consensus on the task type was reached in 89\% of the cases (551 cases with no clear majority).

%evaluation
	\item Using the labelled data we trained a multi-class SVM classifier for the 6 different task types and evaluated its quality with 10-folds cross validation over the labelled dataset. Overall, the classifier obtained Precision of 0.895, Recall of 0.899, and F-Measure of 0.895. Most classifier errors (i.e., 66 instances) were performed by incorrectly classifying IA instances as CC.

	\item Using a classifier trained over the entire labelled data, we performed a large scale classification of the 2.5M HITs in our collection. This allow us to study the evolution of task type on the market.

\end{itemize}


\subsection{Which Task Types Were Popular Over Time?}
%!TEX root = ../dynamics.tex
\section{Predicting the Platform Throughput}\label{sec:Throughput}
\begin{itemize}

	\item ML tasks: classification and regression

	\item ML results: accuracy and error

	\item Best features

\end{itemize}


\begin{figure}[htbp]
	\centering
		\includegraphics[width=0.5\textwidth]{accuracy.png}
	\caption{Accuracy}
	\label{fig:accuracy}
\end{figure}

\begin{figure}[htbp]
	\centering
		\includegraphics[width=0.5\textwidth]{falsepred.png}
	\caption{falsepred}
	\label{fig:falsepred}
\end{figure}


\begin{figure}[htbp]
	\centering
		\includegraphics[width=0.5\textwidth]{mae.png}
	\caption{mae}
	\label{fig:mae}
\end{figure}


\begin{figure}[htbp]
	\centering
		\includegraphics[width=0.5\textwidth]{mse.png}
	\caption{mse}
	\label{fig:mse}
\end{figure}
%!TEX root = ../dynamics.tex
\section{Market Analysis}
\label{sec:market}
Finally, we study the \emph{demand vs supply} on the Amazon MTurk marketplace.
In this case, $Demand$ is defined as the number of new tasks \emph{published} on the platform by the requesters.
In addition we compute the average reward of the posted tasks.
Conversely, $Supply$ is defined as the workforce that the crowd is providing concretized as the number of tasks that got \emph{completed} in a given time window by the workers.
Again, we compute the average reward of the completed tasks.

An initial behavior  that we observe is  the strong weekly periodicity that the demand (requesters) exhibits reflected by the autocorrelation that we compute on the number of available HITs as reported by Amazon Mturk (See Figure \ref{fig:autocorrelation1}).
\begin{figure}[tb]
	\centering
		\includegraphics[width=0.48\textwidth]{figures/autocorrelation_plot}
	\caption{Autocorrelation of the number of HITs available on MTurk time series.
There is a significant correlation that lasts 7-10 days (see left part of the graph 0-250 Hours).}
	\label{fig:autocorrelation1}
\end{figure}

\subsection{New Tasks Attract New Workers}
\begin{figure}[tb]
	\centering
		\includegraphics[width=0.48\textwidth]{figures/scattermatrix}
	\caption{Scatter matrix comparing the different parameters in the demand and supply.}
	\label{fig:scatter_matrix}
\end{figure}
\begin{figure}[tb]
	\centering
		\includegraphics[width=0.48\textwidth]{figures/supply_demand}
	\caption{Demand and supply versus price.}
	\label{fig:dsup}
\end{figure}
In Figure \ref{fig:scatter_matrix} we compare  the following variables: number of HITs published on the platform, average reward of such HITs, number of HITs completed, and the average reward of such HITs. An observation  can be made about the correlation between the rewards and the number of HITs completed and those published.
The results suggest that crowd workers are sensitive to newly posted tasks, and that they are constantly monitoring for new and fresh tasks.
It must be noted that some workers may not been actively completing HITs on the platform but rather looking for HITs to work on by reading Web forums where other workers shared HITs. Thus, new HITs attract new workers to the platform.
This supplements our finding in Section \ref{sec:throughput} where we showed that $Start\_time$ is an important feature contributing to the throughput of a batch.

Figure \ref{fig:dsup}(a) shows a the expected relationship between demand and price: The larger the demand the lower the price becomes. On the other hand Figure \ref{fig:dsup}(b)  does not match the typical supply curve (higher prices drive higher supply). Instead, the demand seems to be tied to  the supply (i.e., the number of new HITs posted). A possible explanation to this effect is the lack of biding (or negotiation) means that the workers can leverage to drive prices up.

\subsection{New Workers Complete More Tasks}
\begin{figure}[tb]
	\centering
		\includegraphics[width=0.48\textwidth]{figures/percHitsCompleted}
	\caption{The effect of new arrived HITs on the work  supplied.}
	\label{fig:perc_hits_completed}
\end{figure}
Along the same lines, the results detailed in Figure \ref{fig:perc_hits_completed} indicate that as more HITs are published on the platform, a higher percentage of the available work gets done. 
This clearly indicates that the arrival of new work also attracts new workers
to the market, who also seem to spillover to other tasks (i.e., not just working on the fresh HITs).

\subsection{Weekly Periodicity}
Finally, we computed the weekly moving average convergence/divergence (MACD) which is shown in Figure \ref{fig:mac}. We then run an autocorrelation to check whether there is some seasonality in the time series. Figure \ref{fig:autocorrelation2} shows that there is a strong weekly seasonality effect.
\begin{figure}[tb]
	\centering
		\includegraphics[width=0.48\textwidth]{figures/mac}
	\caption{Weekly Moving Average Convergence/Divergence (MACD) of the rewards for completed HITs.}
	\label{fig:mac}
\end{figure}
\begin{figure}[tb]
	\centering
		\includegraphics[width=0.48\textwidth]{figures/autocorrelation2}
	\caption{Autocorrelation computed on the weekly MACD. Again we see a weekly periodicity (0-250 Hours).}
	\label{fig:autocorrelation2}
\end{figure}

%!TEX root = ../dynamics.tex
\section{Discussion}
\label{sec:discuss}

The main findings of our analysis are as follows:
\begin{itemize}[noitemsep,topsep=0pt,parsep=0pt,partopsep=0pt]
	\item Tasks related to audio transcription have been gaining momentum in the last years and are today the most popular tasks on \amt{}.
	\item The popularity of Content Access HITs has decreased over time. Surveys are however becoming more popular over time especially in the US.
	\item While most HITs do not require country-specific workers, most of such HITs require US-based workers.
	\item HITs that are exclusively asking for workers based in India have strongly decreased over time.
	\item Surveys are the most popular type of HITs for US-based workers.
	\item The most frequent HIT reward value has increased over time, and reaches \$0.05 in 2014.
	\item New requesters constantly join \amt{}, making the total number of active requesters and the available reward increase over time.
	\item The average HIT batch size is constant over time; however, very large batches have recently started to appear on the platform.
	\item The throughput of HIT batches can best be predicted based on the number of HITs available in the batch (i.e., its size) and its freshness.
	\item New HITs attract new workers to the platform.
	\item New workers arriving to the platform complete both fresh and old HITs.
	\item There is a weekly seasonality effect in the amount of rewards assigned to workers and HITs available.
\end{itemize}


A possible explanation for this is the lack of bidding (or negotiation) mechanisms available to the workers; such mechanisms are not available on the current platform though they could help workers drive the prices up. 

It must be noted that some workers might not be actively completing HITs on the platform but rather looking for HITs to work on by reading Web forums where other workers shared HITs. In that sense, new HITs attract new workers to the platform. This supplements our finding from Section \ref{sec:throughput}, where we showed that $start\_time$ is an important feature contributing to the pace of a batch.


This periodicity can either be rooted in the routine of workers
%!TEX root = ../dynamics.tex
\section{Conclusions}\label{sec:conc}

In this paper, we presented an analysis of the evolution of micro-task crowdsourcing over the past five years.
We studied data collected from a popular micro-task crowdsourcing platform: \amt{}
and we analyzed a number of key dimensions of the platform, including: topic, task type,  reward evolution, platform throughput, supply and demand curves. The results of our analysis can serve as a starting point for improving existing crowdsourcing platforms and to optimize the overall efficiency and effectiveness of human computation systems. The evidence presented above  indicate how requesters should use crowdsourcing platforms to obtain the best out of them: Engage with workers and publish large volumes of HITs at specific points in time. 

%For this purpose we created a system that can support requesters in publishing their HITs optimizing for throughput\footnote{\url{http://xi-lab.github.io/mturk-mrkt/}}.

\gd{added some future work in the following. can be removed.}
Future research based on this work will look at different directions. On one hand, novel micro-task crowdsourcing platforms need to be re-designed based on the findings identified in this work such as the need for specific task type support (e.g., audio transcription) and the need to reserve certain worker types.
% 
Additionally, analysis that look at specific segments can provide additional understanding of the micro-task crowdsourcing universe. Examples include a per-requester analysis of publishing behavior evolution rather then looking at the entire market evolution as done in this work. This can lead to a definition of different classes of requesters which will provide a better understanding of requester requirements (e.g., support for specific task types like surveys or support in identifying the most appropriate workers in the crowd). Similarly, a worker-centered analysis could provide evidence of different existing crowd worker classes (e.g., full-time vs casual workers; workers specializing on specific task types as compared to generalists  who are willing to complete any task available).

\appendix

\section{Features Used in the Machine Learning}
I
\begin{table}[h]
\begin{tabular}{|l|l|l|}
\hline
\textbf{Field} & \textbf{Datatype} & \textbf{Details}\\ \hline\hline
Time     &  DateTime & The time bucket of the Crawling\\ \hline
Hour      & Int    &  Hour of the day   \\ \hline
Weekday      & Int   &  Day of the week (Monday=0)  \\ \hline
requester\_id     & String &   ID of the requester       \\ \hline
location      & String &  The requested worker's Location (e.g. US)       \\ \hline
totalapproved    & Int &  \#total approved HITs      \\ \hline
approvalrate      & Int&   \%of worker's approvals       \\ \hline
master     &  Boolean    &  Worker is a master  \\ \hline
qualification    &Boolean  &   Requested qualification test       \\ \hline
bonus      &Boolean    &  Title/Description mentioning a Bonus    \\ \hline
Invite      & Boolean     &    Title/Description mentioning a specific worker\\ \hline
titlelength      & Int&   String size of the the title       \\ \hline
desclength    & Int &  String size of the description      \\ \hline
keywords      & String & Keywords (space separated)        \\ \hline
ageminutes      &  Int  &  Age since the Batch was posted (minutes)   \\ \hline
leftminutes      & Int  & Time left before expiration (minutes) \\ \hline
time\_alloted    & Int &  Time allotted per task   \\ \hline
totalbatchs      & Int  & Total number of concurrent Batches      \\ \hline
totalhis      & Int    &  Total number of HITs on mturk   \\ \hline
reward    & Double      & HIT Reward in USD  \\ \hline
prev      & Int     &  Number of available HITs in the previous Bucket  \\ \hline
\textbf{Diff}      & Int &   \textbf{Target Class}: Number of HITs completed from last Time \\ \hline
\end{tabular}
\end{table}

\bigskip
%\small
\bibliographystyle{abbrv}
\bibliography{crowd}



\end{document}

