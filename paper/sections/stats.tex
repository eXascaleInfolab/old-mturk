%!TEX root = ../dynamics.tex
\section{The Evolution of Amazon MTurk: 2009 to 2014}\label{sec:stats}

\subsection{Crowdsourcing Platform Dataset}
\label{sec:tracker}
Over the past five years, we have periodically collected data about HITs being published on \amt\ .
and is available at \url{http://mturk-tracker.com/}. In this work we consider hourly aggregated data that includes the available batches and their metadata (tittle, description, rewards, required qualifications etc), in addition to their progress overtime, that is the temporal variation of HITs available. In fact one of the main metrics that we investigate, especially in section \ref{sec:throughput}, is the throughput of a batch, or how many hits it gets completed between two observations. Note that the tracker acts only as a periodic observer and do not reflect fine grained information that only \amt\ might possess. We believe however that it captures enough information to perform long term trend analysis.\\
\subsection{A Data-driven Analysis of Platform Evolution}
First, we try to identify some trends obtained from aggregated information over years, keywords or countries.  Each of the following analyses is available as an interactive visualization of the historical data on \url{http://xi-lab.github.io/mturk-mrkt/}.
\paragraph{Popular Topics  Over Time}
First we want to understand how different topics have been addressed by means of micro-task crowdsourcing over time.
In order to do such analysis we look at the tags associated with published HITs. We observe the evolution of tag popularity and associated reward on \amt\ . 
%plot explanation
Figure \ref{fig:tagEvolution} shows such behavior. Each point in the plot represent a tag associated to HITs with its frequency (i.e., number of HITs with such tag) on the x asis, its average reward of HITs with this tag on the y axis in a certain year. The path connecting data points indicates the time evolution, starting in 2009, with one point representing the tag usage over one year.
% observations
We can observe that `audio' and `transcription' tags (i.e., blue and red path from left to right) have substantially increased in frequency over time being among the most popular tags in the last two years and are paid more than \$1 on average.
HITs with the `video' tag has also increased in number with a reward that has reached a peak in 2012 and decreased after that.
HITs tagged as `categorization' have been paid constantly in the range \$0.10-\$0.30 on average, except in 2009 where they were rewarded less than \$0.1.
HITs tagged as `tweet' have not increased in number but have been paid more over years reaching \$0.9 in 2014: This can be explained by more complex tasks being requested to workers: for example, from sentiment classification to writing a tweet.

\begin{figure}[htbp]
	\centering
		\includegraphics[width=0.5\textwidth]{figures/tagEvolution}
	\caption{tagEvolution}
	\label{fig:tagEvolution}
\end{figure}

\paragraph{Country Preference by Requesters Over Time}
Figure \ref{fig:country} shows the requirements of requesters with respect to which countries they need workers from. The left part of Figure \ref{fig:country} shows that most HITs are to be completed exclusively by workers located in US, India, or Canada. The right part of Figure \ref{fig:country} shows the evolution over time of the country requirement phenomena.
The plot shows the number of HITs with a certain country requirement (on the y axis) and its time evolution (on the x axis) with yearly steps. The size of the data point indicates the total reward associated to those HITs.
We can see that US-only HITs dominate both in terms of number as well as of reward associated to them. 
Interestingly, we notice how over time HITs uniquely for workers based in India have been decreasing. 
While HITs for workers based in Canada have been increasing over time being in 2014 more than those for workers based in India, we see that the reward associated to them is less as compared to the budget for Indian-only HITs.
As of 2014, both HITs for workers based in Canada or UK are more numerous that those for workers based in India.
Overall, 88.5\% of the batches that were posted in the considered time period didn't require any specific worker location, and 86\% of those who did requested US based workers.
\begin{figure*}[htbp]
	\centering
		\includegraphics[width=0.43\textwidth]{figures/map}
		\includegraphics[width=0.43\textwidth]{figures/countriesTime}
	\caption{HITs with specific country requirements. On the left, the countries with most HITs dedicated to them. On the right, the time evolution (x-axis) of country specific HITs with volume (y-axis) and reward information (size of data point).}
	\label{fig:country}
\end{figure*}

Figure \ref{fig:keyword_loc} shows the top keywords attached to HITs restricted to specific locations.
\begin{figure}[htbp]
	\centering
		\includegraphics[width=0.5\textwidth]{figures/keywords_location}
	\caption{Keywords for HITs restricted to specific countries.}
	\label{fig:keyword_loc}
\end{figure}
We can see that the most popular keywords (i.e., `audio' and `transcription') do not require country-specific workers. This can be explained by the fact that restricting on workers from a single country such large amount of HITs would make their completion much slower. We also not that US-only HITs are most commonly tagged with `survey'.

% \subsection{Why Did Certain Requesters Quit?}
% \subsection{Does reputation Improve Throughput?}

\paragraph{HIT Reward Analysis}
Figure \ref{fig:reward_year} shows most frequent rewards assigned to HITs over time\footnote{Data for 2014 has been omitted as not comparable with other year values.}. We can see that while in 2011 the most popular reward was \$0.01, more recently HITs paid \$0.05 are more frequent. This can be explained by both how workers search for HITs on \amt\ and by the \amt fee scheme. Requesters now prefer to publish more complex HITs possibly with multiple questions in it and grant an higher reward: This attracts also those workers who are not willing to complete a HIT for a little pay and reduces the fee paid to \amt which are based on the number of HITs published on the platform.

\begin{figure}[htbp]
	\centering
		\includegraphics[width=0.5\textwidth]{figures/reward_year}
	\caption{Reward \gd{can we add 0.10?}}
	\label{fig:reward_year}
\end{figure}



\paragraph{Requester Analysis}
In order to be sustainable, a crowdsourcing platform needs to retain requesters over time or get new requesters to replace those who do not publish HITs anymore. Figure \ref{fig:requesters_reward} shows the number of new requesters who used \amt and the overall number of active requesters at a certain point in time. We can observe an increasing number of active requesters over time and a constant number of new requesters who join the platform (at a rate of 1000/month over the last two years).

\begin{figure}[htbp]
	\centering
		\includegraphics[width=0.5\textwidth]{figures/requesters_reward}
	\caption{Requester activity and total reward on the platform over time. The line over the data shows local polynomial regression fitting \cite{cleveland1992local}.}
	\label{fig:requesters_reward}
\end{figure}

To understand the financial health of the crowdsourcing platform is also interesting to see the overall reward amount published on the platform as its revenues are computed in function of the HIT rewards. From the bottom part of Figure \ref{fig:requesters_reward} we can see observe a linear increase in the total reward for HITs on the platform. Interestingly, we can also observe some seasonality effects over years with October being the month with highest total reward and January or February being the month with minimum total reward.


\paragraph{HIT Batch Size Analysis}
When a lot of data needs to be crowdsourced (e.g., many images to tag), multiple similar HITs can be published together. We define a batch of HITs as such set of HITs published by a requester at a certain point in time. Figure \ref{fig:batch_size} shows how over time batch size has changed.
We can see that while on average batches have been getting smaller, in 2014 very large batches have appeared on \amt indicating a clear business interest and scalability demand from micro-task crowdsourcing.
s
\begin{figure}[htbp]
	\centering
		\includegraphics[width=0.5\textwidth]{figures/batch_size}
	\caption{Batch Sizes. The line over the data shows local polynomial regression fitting \cite{cleveland1992local}.}
	\label{fig:batch_size}
\end{figure}


