%!TEX root = ../dynamics.tex
\section{Introduction}\label{sec:intro}
The micro-task crowdsourcing market has seen a rapid growth in the last five years. This is correlated with the fact that more data is available to enterprises and that data is an asset within business processes. While data is available, its quality is not always high and manual processing is often necessary. To this end, outsourcing  data processing tasks  like image tagging, audio transcription, translation, etc. to a large crowd of individuals has become popular.

%micro-task Crowdsourcing platform
To perform such Human Intelligence Tasks (HITs) crowdsourcing platforms have been developed. Such platforms serve as place where the crowd (i.e., people willing to perform small tasks in exchange of a small monetary reward, also know as \emph{workers}) and work providers (i.e., also known as \emph{requesters}) meet. \gd{describe crowdsourcing process of publishing tasks, completing task, getting results, rewarding}

%
In this paper we analyze the evolution of a very popular micro-task crowdsourcing platform over the last five years and propose techniques to support requesters in posting HITs to the market in an effective way. Using features derived from a large-scale analysis of platform logs, we propose methods to predict the throughput of the crowdsourcing platform for a batch of HITs published by a certain requester at a certain point in time. This prediction is based on different features including current platform load, requester reputation, popular task types, etc.

\gd{summarize main findings}


In summary, the main contributions of this paper are:
\begin{itemize}

	\item An analysis of the evolution of a popular micro-task crowdsourcing market looking at dimensions like topics, task types, reward, worker location.

	\item A large-scale classification of 2.5M HITs published on Amazon MTurk.

	\item An effective model to predict the completion speed of a batch published on Amazon MTurk at a certain point by a certain requester.

	\item An online service with evolution statistics and services for requesters to better design their HIT metadata (e.g., keywords, reward, etc.)

\end{itemize}


The rest of the paper is structured as follows.
In Section \ref{sec:relwork} we overview recent work on micro-task crowdsourcing specifically focusing on works improving efficiency of crowdsourcing systems.