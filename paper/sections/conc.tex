%!TEX root = ../dynamics.tex
\section{Conclusions}\label{sec:conc}

In this paper, we presented an analysis of the evolution of micro-task crowdsourcing over the past five years.
We studied data collected from a popular micro-task crowdsourcing platform: \amt{}
and we analyzed a number of key dimensions of the platform, including: topic, task type,  reward evolution, platform throughput, supply and demand curves. The results of our analysis can serve as a starting point for improving existing crowdsourcing platforms and to optimize the overall efficiency and effectiveness of human computation systems. The evidence presented above  indicate how requesters should use crowdsourcing platforms to obtain the best out of them: Engage with workers and publish large volumes of HITs at specific points in time. 

%For this purpose we created a system that can support requesters in publishing their HITs optimizing for throughput\footnote{\url{http://xi-lab.github.io/mturk-mrkt/}}.

\gd{added some future work in the following. can be removed.}
Future research based on this work will look at different directions. On one hand, novel micro-task crowdsourcing platforms need to be re-designed based on the findings identified in this work such as the need for specific task type support (e.g., audio transcription) and the need to reserve certain worker types.
% 
Additionally, analysis that look at specific segments can provide additional understanding of the micro-task crowdsourcing universe. Examples include a per-requester analysis of publishing behavior evolution rather then looking at the entire market evolution as done in this work. This can lead to a definition of different classes of requesters which will provide a better understanding of requester requirements (e.g., support for specific task types like surveys or support in identifying the most appropriate workers in the crowd). Similarly, a worker-centered analysis could provide evidence of different existing crowd worker classes (e.g., full-time vs casual workers; workers specializing on specific task types as compared to generalists  who are willing to complete any task available).