%!TEX root = ../dynamics.tex
\section{Conclusions}\label{sec:conc}

In this paper, we presented an analysis of the evolution of micro-task crowdsourcing over the past five years.
We studied data collected from a popular micro-task crowdsourcing platform: \amt{}
and we analyzed a number of key dimensions of the platform, including: topic, task type, or reward evolution, platform throughput, supply and demand curves.
The main findings of our analysis are as follows:
\begin{itemize}[noitemsep,topsep=0pt,parsep=0pt,partopsep=0pt]
	\item Tasks related to audio transcription have been gaining momentum in the last years and are today the most popular tasks on \amt{}.
	\item The popularity of Content Access HITs has decreased over time. Surveys are however becoming more popular over time especially in the US.
	\item While most HITs do not require country-specific workers, most of such HITs require US-based workers.
	\item HITs that are exclusively asking for workers based in India have strongly decreased over time.
	\item Surveys are the most popular type of HITs for US-based workers.
	\item The most frequent HIT reward value has increased over time, and reaches \$0.05 in 2014.
	\item New requesters constantly join \amt{}, making the total number of active requesters and the available reward increase over time.
	\item The average HIT batch size is constant over time; however, very large batches have recently started to appear on the platform.
	\item The throughput of HIT batches can best be predicted based on the number of HITs available in the batch (i.e., its size) and its freshness.
	\item New HITs attract new workers to the platform.
	\item New workers arriving to the platform complete both fresh and old HITs.
	\item There is a weekly seasonality effect in the amount of rewards assigned to workers and job completed by them.
\end{itemize}

The results of our analysis can serve as a starting point for improving existing crowdsourcing platforms and to optimized the overall efficiency and effectiveness of human computation systems. The evidence presented above  indicate how requesters should use crowdsourcing platforms to obtain the best out of them: Engage with workers and publish large volumes of HITs at specific points in time. 

%For this purpose we created a system that can support requesters in publishing their HITs optimizing for throughput\footnote{\url{http://xi-lab.github.io/mturk-mrkt/}}.