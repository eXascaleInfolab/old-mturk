%!TEX root = ../dynamics.tex
\section{Conclusions}\label{sec:conc}

In this paper, we presented an analysis of the evolution of micro-task crowdsourcing over the past five years.
We studied data collected from a popular micro-task crowdsourcing platform: \amt{}
and we analyzed a number of key dimensions of the platform, including: topic, task type, or reward evolution, platform throughput, supply and demand curves. The results of our analysis can serve as a starting point for improving existing crowdsourcing platforms and to optimized the overall efficiency and effectiveness of human computation systems. The evidence presented above  indicate how requesters should use crowdsourcing platforms to obtain the best out of them: Engage with workers and publish large volumes of HITs at specific points in time. 

%For this purpose we created a system that can support requesters in publishing their HITs optimizing for throughput\footnote{\url{http://xi-lab.github.io/mturk-mrkt/}}.