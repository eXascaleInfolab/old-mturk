%!TEX root = ../dynamics.tex
\section{Discussion}
\label{sec:discuss}

In this section, we summarize the main findings of our analysis and a present a discussion of our results. We  extracted several trends from the five years dataset, summarized as follows:
\begin{itemize}[noitemsep,topsep=0pt,parsep=0pt,partopsep=0pt]
	\item Tasks related to audio transcription have been gaining momentum in the last years and are today the most popular tasks on \amt{}.
	\item The popularity of Content Access HITs has decreased over time. Surveys are however becoming more popular over time especially in the US.
	\item While most HITs do not require country-specific workers, most of such HITs require US-based workers.
	\item HITs that are exclusively asking for workers based in India have strongly decreased over time.
	\item Surveys are the most popular type of HITs for US-based workers.
	\item The most frequent HIT reward value has increased over time, and reaches \$0.05 in 2014.
	\item New requesters constantly join \amt{}, making the total number of active requesters and the available reward increase over time.
	\item The average HIT batch size has been stable over time; however, very large batches have recently started to appear on the platform.
\end{itemize}

Next, we evaluated the features that influence the progress of a batch of tasks, and their importance over time. Our batch throughput prediction method indicates that the throughput of HIT batches can best be predicted based on the number of HITs available in the batch (i.e., its size) and its freshness.

Finally, we analyzed \amt{} as a marketplace with demand (new HITs arriving) and supply (HITs completed). 
%We saw, for instance, that higher demand does not lead to higher HIT prices. A possible explanation for this is the lack of bidding (or negotiation) mechanisms available to the workers; such mechanisms are not available on the current platforms though they could help workers drive the prices up. 
We also observed strong weekly periodicity in both demand and supply. While requesters might have routine business needs, we can hypothesize that many workers work on \amt{} on a regular basis.