%!TEX root = ../dynamics.tex
\section{Market Analysis}
\label{sec:market}
Finally, we study the demand and supply of Amazon MTurk marketplace. In the following, we define $Demand$ as the number of new tasks published on the platform by the requesters. In addition, we compute the average reward of the tasks that were posted. Conversely, we define $Supply$ as the workforce that the crowd is providing concretized as the number of tasks that got completed in a given time window by the workers. Again, we compute the average reward of the completed tasks.
In this section we use hourly collected data for the time period between 01 Jun 2014 and 15 Oct 2014.

\subsection{Supply vs. Demand.}
First, we check whether supply and demand, in the context of Amazon MTurk marketplace, exhibit the standard behavior and relationship.
Figure \ref{fig:dsup} (right) shows the expected relationship between supply and price: The larger the supply, the lower the price evolves, which is the usual behaviour. On the other hand, Figure \ref{fig:dsup} (left) demand does not match the typical ascending demand curve (``the higher the demand the higher the price evolves''). Instead, the demand seems to be tied to the supply (i.e., the number of new HITs completed).

\begin{figure}[tb]
	\centering
		\includegraphics[width=0.48\textwidth]{figures/ds}
	\caption{The supply and demand for HITs versus the average HIT price.}
	\label{fig:dsup}
\end{figure}

\subsection{Supply Attracts new Workers.} %

Now, we examine how the supply attracts new workers. First, we compare the three following variables: number of HITs published on the platform, number of HITs completed, and the average reward per HIT of both published and completed tasks. The results are depicted in Figure \ref{fig:scatter_matrix}. The apparent correlation between the rewards of the HITs completed and those published, and also among their quantities, gives us a first intuition that the crowd workers are sensitive to newly posted tasks, especially for those with higher prices.

\begin{figure}[tb]
	\centering
		\includegraphics[width=0.48\textwidth]{figures/scatter}
	\caption{Scatter matrix comparing the different parameters in the demand and supply.}
	\label{fig:scatter_matrix}
\end{figure}

Next, we compare the number of HITs arrived with the percent of HITs completed.
The results in Figure \ref{fig:perc_hits_completed} indicate that, as more HITs are published on the platform, a higher percentage of the available work gets done. 
Running a statistical ordinary least square (OLS) regression gives us [intercept = 0.25, slope coeficient=0.05]. Thus, when many HITs arrive to the platform, the overall percentage of HITs completed on the platform tends to slightly increase.
This also indicates that the arrival of new work also attracts new workers to the market, who also seem to spill over to other tasks (i.e., they are not just working on the fresh HITs). 
\begin{figure}[tb]
	\centering
		\includegraphics[width=0.48\textwidth]{figures/percHitsCompleted.pdf}
	\caption{The effect of new arrived HITs on the work  supplied.}
	\label{fig:perc_hits_completed}
\end{figure}

\subsection{Demand and Supply Periodicity}
On the demand side, we often observe some requesters frequently posting new batches of recurrent tasks. Hence, we are interested in the periodicity of such demand in the marketplace and the supply it drives. For that, we consider both the time-series of available HITs and the rewards completed over the period of three months. 

First, we observe that the demand exhibits a strong weekly periodicity, which is reflected by the autocorrelation that we compute the number of available HITs on Amazon Mturk (See Figure \ref{fig:autocorrelation1}). The market seems to have a significant memory that lasts for approximately 7-10 days.

\begin{figure}[tb]
    \centering
    \begin{subfigure}[b]{0.48\textwidth}
        \centering
        \includegraphics[width=\textwidth]{figures/out}
        \caption{HITs Available.}
        \label{fig:y equals x}
    \end{subfigure}
    \hfill
    \begin{subfigure}[b]{0.48\textwidth}
        \centering
        \includegraphics[width=\textwidth]{figures/out1}
        \caption{Autocorrelation onr HITs available.}
        \label{fig:three sin x}
    \end{subfigure}
    \hfill 
    \caption{Autocorrelation of the number of HITs available on MTurk time series. There is a significant correlation that lasts 7-10 days (see left part of the graph 0-250 Hours).}
	\label{fig:autocorrelation1}
\end{figure} 

Conversely, and to check for periodicity in the supply, we compute an autocorrelation on the weekly moving average convergence of the reward attached to completed HITs. Figure \ref{fig:autocorrelation2} shows that there is a strong weekly periodicity effect as we observe high values in the range 0-250 hours.

\begin{figure}[tb]
    \centering
    \begin{subfigure}[b]{0.48\textwidth}
        \centering
        \includegraphics[width=\textwidth]{figures/mac}
        \caption{Weekly moving average on rewards completed.}
        \label{fig:mac}
    \end{subfigure}
    \hfill
    \begin{subfigure}[b]{0.48\textwidth}
        \centering
        \includegraphics[width=\textwidth]{figures/macac}
        \caption{Autocorrelation on rewards completed.}
        \label{fig:macac}
    \end{subfigure}
    \hfill
	\caption{Autocorrelation computed on the weekly rewards moving average. Again we see a weekly periodicity (0-250 Hours).}
	\label{fig:autocorrelation2}
\end{figure}